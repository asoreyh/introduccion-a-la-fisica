\documentclass[a4paper,12pt]{article}
\usepackage[spanish]{babel}
\hyphenation{co-rres-pon-dien-te}
%\usepackage[latin1]{inputenc}
\usepackage[utf8]{inputenc}
\usepackage[T1]{fontenc}
\usepackage{graphicx}
\usepackage[pdftex,colorlinks=true, pdfstartview=FitH, linkcolor=blue,
citecolor=blue, urlcolor=blue, pdfpagemode=UseOutlines, pdfauthor={H. Asorey},
pdftitle={Introducción a la Física - Diagnóstico}, pdfkeywords={dianostico}]{hyperref}
\usepackage[adobe-utopia]{mathdesign}

\hoffset -1.23cm
\textwidth 16.5cm
\voffset -2.0cm
\textheight 26.0cm

%----------------------------------------------------------------
\begin{document}
\begin{center}
{\small{Universidad Industrial de Santander - Escuela de Física}}\\
{\bf{Introducción a la Física (Asorey-Sarmiento-Pinilla)}}\\
\vspace{0.7cm}
Evaluación Diagnóstica -- 2014
\end{center}

\noindent {\bf{Importante: \\ Este examen se realiza sólo por motivos diagnósticos. El resultado no afectará la nota final del curso}}\\

\noindent Opcional: Si usted desea conocer su diagnóstico, por favor indique su nombre: \\

\begin{enumerate}
  \item Factorice los siguientes polinomios
    \begin{enumerate}
      \item $x^2 - 6x +9$:
      \item $24 x^2 + 16 x + 2$:
      \item $81 x^2 - 9 y^2$:
      \item $18 x^4 y^3 + 18 x^4 y^4 + 12 x^3 y^4 + 12 x^2 y^5$
    \end{enumerate}

  \item Calcule
    \begin{enumerate}
      \item
        \begin{equation}
          \frac53 + \frac27 - \frac19 = ? \nonumber
        \end{equation}
      \item
        \begin{equation}
          \left ( \frac x3 \right ) \left ( \frac5x \right ) \left ( \frac85 \right ) = ? \nonumber
        \end{equation}
      \item 
        \begin{equation}
          \frac{\left ( \frac{8}{3}\right )}{\left ( \frac{6}{5}\right )} = ? \nonumber
        \end{equation}
      \item 
        \begin{equation}
          \frac{\frac{4x}{3y^3}}{\frac{12x^2}{9y}} = ? \nonumber 
        \end{equation}
    \end{enumerate}
  \item Resuelva las siguientes ecuaciones:
    \begin{enumerate}
      \item $\frac{3x}{4} + 1 = 7 \frac{(x-2)}{6}$
      \item $x^2 + 2x = 8$
      \item $3(4 - x) + 2x = 9 - 4(x - 2) + 3x + 1$
    \end{enumerate}
  \item Responda: Juan tiene en total $136$ monedas de $\$50$, $\$100$ y $\$200$. ¿Cuánto dinero tiene si las monedas de $\$50$ son la mitad de las de $\$200$, y estas a su vez son el quíntuplo de las monedas de $\$100$?
  \item Calcule: 
    \begin{enumerate}
      \item el volumen de una semiesfera de radio $r=2$\,m
      \item el volumen de un paralelepípedo de base cuadrada de $3$\,cm de base y $5$\,cm de altura
      \item el volumen de un cilindro de $r=1$\,m y altura $h=7$\,m
    \end{enumerate}
  \item Sea un triángulo rectángulo con hipotenusa $h=10$\,cm y uno de los ángulos mide $\alpha=30^\mathrm{o}$. Dibuje el triángulo y calcule la longitud del cateto opuesto y del cateto adyacente.
    
\item Sean $\mathbf{v}_1=(3,-2,1)$, $\mathbf{v}_2=(-2,1,2)$ y $\mathbf{v}_3=(0,0,0)$. Responda:
    \begin{enumerate}
      \item ¿Cuál es la dimensión de los vectores $\mathbf{v}_i$?
      \item Dibuje el vector $\mathbf{v}_2$. Indique la dirección y y el sentido del vector. Luego, calcule su módulo $v_2 \equiv |\mathbf{v}_2|$ 
      \item Obtenga gráfica y analíticamente el vector $\mathbf{v}_s=\mathbf{v}_1 + \mathbf{v}_2$
      \item Obtenga gráfica y analíticamente el vector $\mathbf{v}_r=\mathbf{v}_1 - \mathbf{v}_2$
      \item Calcule el producto escalar $\mathbf{v}_1 \cdot \mathbf{v}_2$ y $\mathbf{v}_1 \cdot \mathbf{v}_3$
    \end{enumerate}
%%%%%%%%%%%%%%%%%%%%%%%%%%%%%%%%
\end{enumerate}
\end{document}
%%%%
