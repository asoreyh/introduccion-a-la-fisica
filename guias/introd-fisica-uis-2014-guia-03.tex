\documentclass[a4paper,11pt]{article}
\usepackage[spanish]{babel}
\hyphenation{co-rres-pon-dien-te}
%\usepackage[latin1]{inputenc}
\usepackage[utf8]{inputenc}
\usepackage[T1]{fontenc}
\usepackage{graphicx}
\usepackage{amsmath}

\usepackage[pdftex,colorlinks=true, pdfstartview=FitH, linkcolor=blue, citecolor=blue, urlcolor=blue, pdfpagemode=UseOutlines, pdfauthor={H. Asorey}, pdftitle={Introducción a la Física - Guía 03} pdfkeywords={big bang}]{hyperref}
\usepackage[adobe-utopia]{mathdesign}

\hoffset -1.23cm
\textwidth 16.5cm
\voffset -2.0cm
\textheight 26.0cm

\begin{document}
\begin{center}
  {\small{Universidad Industrial de Santander - Escuela de Física}}\\
  {\bf{Introducción a la Física 2014 (Asorey-Sarmiento-Pinilla)}}\\
  \vspace{0.3cm}
  Guía 03: Trabajando con la expansión del Universo: El Big Bang
\end{center}

\renewcommand{\labelenumi}{\arabic{enumi})}
\renewcommand{\labelenumii}{\arabic{enumii})}

\subsubsection*{Modalidad de Entrega}
Deberán realizar un informe grupal de no más de tres páginas en \LaTeX, identificando claramente los miembros del grupo. Dado que nos interesa que empiecen a trabajar en \LaTeX, en este informe la presentación final del informe no será evaluada (salvo el cumplimiento de los lineamientos expresados debajo), sólo los conceptos vertidos y los análisis realizados. En todos los casos, utilicen todos los materiales que consideren necesarios para justificar sus respuestas, {\bf{citando}} correctamente las fuentes utilizadas.

Recuerde los siguientes lineamientos mínimos para un informe:
\begin{itemize}
  \item El título debe capturar la atención de un posible lector
  \item Es importante identificar correctamente a los autores del trabajo
  \item El informe debe tener un resumen corto que explique las principales características y las conclusiones del trabajo realizado
  \item Las figuras deben poseer epígrafes (pie de gráficas). Estos deben ser autocontenidos: con sólo leer el epígrafe el lector debe ser capaz de entender las gráficas sin necesidad de leer el texto. 
  \item Las figuras deben ser explicadas y referenciadas en el texto del informe.
  \item Las unidades se escriben fuera del entorno matemático: 
  \begin{itemize}
    \item {\bf{incorrecto}}: ``\ldots la distancia medida fue de $1.5 mts$ \ldots'' (se obtuvo así: \verb|$1.5 mts$|. Notar además que ``mts'' no es la abreviatura de metros en el sistema internacional).
    \item {\bf{correcto}}: ``\ldots la distancia medida fue de $1.5$\,m \ldots'' (se obtuvo así: \verb|$1.5$\,m|).
\end{itemize}
  \item Un trabajo no puede tener faltas ni ortográficas ni gramaticales. Cuando termine de escribir, lea el texto en voz alta y corrija aquellas frases que necesiten serlo.
\end{itemize}

\subsubsection*{Preguntas para pensar y responder (lista no excluyente)}

A partir de los datos recogidos en clase midiendo las distancias de alejamiento de los puntos sobre la superficie del globo, responda:

\begin{enumerate}
  \item Tomando como base el punto elegido, observe las diferencias en la evolución temporal de las distancias entre los puntos cercanos y los puntos lejanos. ¿que puede concluir de este análisis? Utilice las gráficas de las mediciones realizadas para justificar su respuesta\footnote{Puede utilizar cualquier software para analizar los datos, como por ejemplo, {\emph{Excel}} o {\emph{Calc}} de {\emph{LibreOffice}} incluido en la máquina virtual de la materia para realizar los análisis.}.
  \item A partir de las mediciones realizadas, ¿es posible concluir que el punto que usted eligió corresponde a un verdadero ``centro'' en la superficie del globo? ¿qué hubiera pasado si hubieran elegido algún otro punto con las conclusiones del estudio?
  \item ¿Cómo evoluciona la longitud de la onda con la expansión del globo? Utilice un gráfico para ilustrar su explicación.
  \item Las mediciones realizadas permiten estimar los radios del globo en cada instante de medición
    \begin{enumerate}
      \item Explique de que forma la distancia entre dos puntos sobre la superficie de una esfera se relacionan con el radio del globo
      \item Utilice lo anterior para obtener una aproximación de la densidad superficial de puntos sobre el globo en cada instante
      \item Haga un gráfico de la evolución temporal de la densidad de puntos sobre el globo.
    \end{enumerate}
\end{enumerate}
\end{document}
