\documentclass[a4paper,12pt]{article}
\usepackage[spanish]{babel}
\hyphenation{co-rres-pon-dien-te}
%\usepackage[latin1]{inputenc}
\usepackage[utf8]{inputenc}
\usepackage[T1]{fontenc}
\usepackage{graphicx}
\usepackage{amsmath}

\usepackage[pdftex,colorlinks=true, pdfstartview=FitH, linkcolor=blue, citecolor=blue, urlcolor=blue, pdfpagemode=UseOutlines, pdfauthor={H. Asorey}, pdftitle={Introducción a la Física - Parcial}, pdfkeywords={Parcial}]{hyperref}
\usepackage[adobe-utopia]{mathdesign}

\hoffset -2.00cm
\textwidth 17.5cm
\voffset -2.0cm
\textheight 26.0cm

\begin{document}
\begin{center}
  {\small{Universidad Industrial de Santander - Escuela de Física}}\\
  {\bf{Introducción a la Física (Asorey-Sarmiento-Pinilla)}}\\
  Parcial Integrador 2014
\end{center}

\renewcommand{\labelenumi}{\arabic{enumi})}
\renewcommand{\labelenumii}{\arabic{enumii})}

\noindent {\small{Lea atenta y cuidadosamente todos los problemas antes de proceder al cálculo de los mismos. Numere y ponga su nombre en todas las páginas que entrega.}}

\begin{enumerate}

  \item {\bf{Guia 04a: Cinemática}} 

    Un auto parte del reposo con una aceleración de $|\vec a|=1$\,m\,s$^{-2}$,
    la cuál se mantiene constante durante $1$\,s. Transcurrido ese tiempo, se
    apaga el motor y el auto desacelera, debido a la fricción, durante $10$\,s
    con una desaceleración promedio $|\vec a|=5$\,cm\,s$^{-2}$. Entonces se
    aplican los frenos y el auto se detiene luego de otros $5$\,s adicionales.
    Calcular la distancia total recorrida por el auto. Hacer los gráficos de
    $x(t)$, $v(t)$ y $a(t)$ como función del tiempo $t$

  \item {\bf{Guia 04b: Energía}}

    El Principito ($m=40$\,kg) vive en un planeta pequeño, el asteroide B612.
    Supongamos que posee un radio $R = 1$\,km con una densidad igual a la de la
    Tierra ($d = 5.5$\,g\,cm$^{-3}$). Con esto, la masa del asteroide queda
    $M=2.3\times 10^13$\,kg. Calcule:
    \begin{enumerate}
      \item el valor de $g$ y el peso del Principito en B612;
      \item la velocidad máxima a la cual el Principito puede caminar sin
        riesgo de abandonar el planeta para siempre (velocidad de escape).
    \end{enumerate}

  \item {\bf{Guia 5: Energía}} 

    Imagine dos cuerpos de masas $m_1$ y $m_2=3 m_1$ que se encuentran en
    reposo ($u_1=u_2=0$). Ambos cuerpos están unidos por un resorte de masa
    despreciable y cuya constante elástica vale $k=1000$\,N\,m$^{-1}$. El
    resorte está comprimido $\Delta x=0.5$\,m respecto de su posición de
    equilibrio. Una vez liberado el resorte, encuentre la relación entre las
    velocidades $\vec v_1$ y $\vec v_2$ de cada cuerpo. Luego calcule dichas
    velocidades y, proponiendo algún valor para la masa $m_1$, calcule la
    energía cinética de cada cuerpo.

  \item {\bf{Guia 5: Kepler}}

    Un nuevo cometa de masa $m=10^{12}$\,kg fue descubierto en el sistema
    solar. Luego de algunas mediciones, se supo que su órbita es elíptica y el
    perihelio está situado a sólo $10^6$ km del Sol.
    
    \begin{enumerate} 
      \item Calcule la distancia al Sol del afelio sabiendo que el período es
        de 10 años.
      \item ¿Cuáles es el valor de la energía potencial en el perihelio y en el
        afelio?
      \item Usando la segunda ley de Kepler, calcule la relación entre las
        energías cinéticas en el afelio y en el perihelio (ayuda: suponga que
        las áreas barridas son triangulares, $A=\frac{1}{2} b \times h$).
    \end{enumerate}


  \item {\bf{Guia 6: Electrostática}}

    Trabajemos con el sistema Tierra-Luna. Suponiendo que transferimos a la
    Tierra y a la Luna la misma cantidad de carga positiva $Q$, calcule el
    valor de $Q$ para que la fuerza de repulsión eléctrica entre ambos cuerpos
    iguale a la fuerza de atracción gravitatoria entre los mismos. Sabiendo que
    la carga de un protón es $p=1.602\times10^{-19}$\,C y que el número de
    Avogadro es $N_A=6.022\times 10^{23}$, diga cuantos moles de protones son
    necesarios para alcanzar el valor de $Q$.

  \item {\bf{Guia 6: Electrostática - Opcional}}

    En el plano $z=0$, dibuje el campo eléctrico producido por una carga
    puntual situada en el orígen

\end{enumerate}
\end{document}
%%%%
