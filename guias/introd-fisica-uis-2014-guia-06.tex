\documentclass[a4paper,12pt]{article}
\usepackage[spanish]{babel}
\hyphenation{co-rres-pon-dien-te}
%\usepackage[latin1]{inputenc}
\usepackage[utf8]{inputenc}
\usepackage[T1]{fontenc}
\usepackage{graphicx}
\usepackage{amsmath}

\usepackage[pdftex,colorlinks=true, pdfstartview=FitH, linkcolor=blue, citecolor=blue, urlcolor=blue, pdfpagemode=UseOutlines, pdfauthor={H. Asorey}, pdftitle={Introducción a la Física - Guía 06}, pdfkeywords={Electrostática}]{hyperref}
\usepackage[adobe-utopia]{mathdesign}

\hoffset -1.23cm
\textwidth 16.5cm
\voffset -2.0cm
\textheight 26.0cm

\begin{document}
\begin{center}
  {\small{Universidad Industrial de Santander - Escuela de Física}}\\
  {\bf{Introducción a la Física (Asorey-Sarmiento-Pinilla)}}\\
  \vspace{0.4cm}
  Guía 06: Electrostática\\ 2014
\end{center}

\renewcommand{\labelenumi}{\arabic{enumi})}
\renewcommand{\labelenumii}{\arabic{enumii})}

\section*{Modalidad de Entrega}

\begin{itemize}
  \item Lea atenta y cuidadosamente todos los problemas antes de proceder al cálculo de los mismos.
  \item Modalidad de trabajo: grupal, en grupos con un mínimo de dos (2) y un máximo de tres (3) personas por grupo. No se admitirán trabajos individuales ni de grupos con más de tres integrantes. 
  \item El trabajo será entregado en el aula 3-7 del CENTIC o en las oficinas del Grupo Halley, Ciencias Humanas Of. 504, en la fecha de vencimiento, teniendo en cuenta lo siguiente:
  \begin{itemize}
    \item la resolución de al menos uno de los ejercicios (a elección de cada grupo) deberá ser entregada en un \texttt{pdf} obtenido utilizando \LaTeX.
    \item la resolución del resto de los ejercicios puede ser realizada ``a mano''
  \end{itemize}
  \item Para esta entrega, valen todas las indicaciones dadas para la entrega de las guías 3, 4 y 5.
  \item El cumplimiento de todas las indicaciones será tenido en cuenta.
  \item Y otra vez, lea atenta y cuidadosamente todos los problemas antes de proceder al cálculo de los mismos.
  \item {\Large{\bf{Fecha límite de entrega: Miércoles 27/Agosto/2014 a las 11:59:59.}}}
\end{itemize}

\section*{Ejercicios}

\begin{enumerate}

\item {\bf{Campo gravitatorio}} %23

  El campo gravitatorio es el campo vectorial producido por la presencia de una
masa o una distribución de masa. En la aproximación de masa puntual, el campo
gravitatorio de un cuerpo de masa $M$ situada en el origen es: \[
\vec{g}(\vec{r}) = \left (\frac{GM}{|\vec{r}|^2} \right )
\hat{r},\] donde $\vec{r}$ es el punto del espacio donde se está
calculando el valor del campo, y por ende $\hat{r}$ es el
correspondiente vector unitario, y $G$ es la constante de Newton.

Calcule el vector campo gravitatorio sobre la superficie de la Tierra, y a una
altura de $1000$\,km sobre la superficie de la misma.

\item {\bf{Sistema Tierra-Luna}} 
 
Repitiendo las cuentas realizadas en clase, calcule la distancia
$d'$, medida desde el centro de la Tierra, a la cual se anula el campo
gravitatorio del Sistema Tierra-Luna. Imagine que situamos el origen de
coordenadas en el centro de la Tierra, de manera que el eje $x$ coincida con la
dirección Tierra-Luna. Recuerde que la distancia Tierra-Luna es
$d_{TL}=3.84\times10^8$\,m.

Luego calcule el vector campo gravitatorio Lunar, el campo gravitatorio
Terrestre y, utilizando el principio de superposición, el campo gravitatorio del
Sistema Tierra-Luna en los siguientes puntos del espacio:

\begin{enumerate}
\item $\vec{r}=(R_\mathrm{Tierra};0;0)$;
\item $\vec{r}=(0;R_\mathrm{Tierra};0)$;
\item $\vec{r}=(-R_\mathrm{Tierra};0;0)$;
\item $\vec{r}=(d_{TL}/2;0;0)$;
\item $\vec{r}=(d';0;0)$;
\item $\vec{r}=(d';10^7;0)$;
\item $\vec{r}=(d_{TL}/2;0;0)$;
\item $\vec{r}=(d_{TL}-R_L;0;0)$;
\item $\vec{r}=(d_{TL}+R_L;0;0)$;
\end{enumerate} 

\item {\bf{Sistema Sol-Júpiter}}

Los cuerpos más masivos del Sistema Solar son, por mucho, el Sol y el planeta
Júpiter. Estudiaremos el Sistema Sol-Júpiter despreciando la existencia de
todos los demás cuerpos del Sistema Solar. 

Suponga entonces que el Sol se encuentra en el origen, y que la dirección del
eje $x$ coincide con la dirección que une al Sol con Júpiter. Midiendo las
distancias en unidades astronómicas ($1$\,UA=$1.5\times10^{11}$\,m), la
posición del Sol será entonces $r_{\mathrm{Sol}} = (0; 0; 0)$ y la posición de Júpiter
$r_\mathrm{Jup} = (5,143; 0; 0)$\,UA. 

Entonces, 
\begin{enumerate}
  \item encuentre el punto del espacio donde el campo gravitatorio total del sistema Sol-Júpiter se anula. ¿Qué objetos del sistema solar se encuentra en esa región del espacio? Aventure una posible explicación para la existencia de esos objetos.
  \item calcule el vector campo gravitatorio Solar, el campo gravitatorio de Júpiter y, utilizando el principio de superposición, el campo gravitatorio del Sistema Sol-Júpiter en la posición de la Tierra, $r_\mathrm{Tierra} = (1; 0; 0)$\,UA
\end{enumerate}

\item {\bf{Gravedad versus Electrostática I}} %26

Imagine dos cuerpos de masas $m_1=m_2=1$\,kg, que se encuentran a una distancia
$|\vec{r}|=1$\,m. Los cuerpos poseen una carga $q_1=q_2=1$\,C. Calcule la
relación entre las energías potenciales gravitatoria $E_g= -G m_1 m_2 /
|\vec{r}|$ y eléctrica $E_e= k_e q_1 q_2 / |\vec{r}|$. Luego, calcule el
valor del potencial eléctrico $V(\vec{r})$ en el punto medio entre las dos
cargas.

\item {\bf{Rayos y centellas}}

Se estima que la corriente eléctrica durante una descarga atmosférica (rayo
eléctrico) puede llegar a $120$\,kA. Sabiendo que la misma dura aproximadamente
$3$\,ms, calcule la cantidad de carga que se transfiere desde la nube hacia la
Tierra durante la descarga. Calcule el valor del potencial eléctrico sobre la
superficie de la Tierra si toda esa carga se distribuye en forma puntual sobre
la base de la nube de tormenta, a una altura de $2000$\,m sobre la superficie.

\item {\bf{Gravedad versus Electrostática II}}

Trabajemos con el sistema Tierra-Luna. Suponiendo
que transferimos a la Tierra y a la Luna la misma cantidad de carga positiva $Q$,
calcule el valor de $Q$ para que la fuerza de repulsión eléctrica entre ambos
cuerpos iguale a la fuerza de atracción gravitatoria entre los mismos. Sabiendo
que la carga de un protón es $p=1.602\times10^{-19}$\,C y que el número de
Avogadro es $N_A=6.022\times
10^{23}$, diga cuantos moles de protones son necesarios para alcanzar el
valor de $Q$.

\item {\bf{La famosa carga puntual}}

Imagine una carga fuente $Q=-1$\,C situada en el origen, y sea un cuerpo de
masa $m=1$\,kg que tiene una carga $q=10^{-9}$\,C situado en la posición
$\vec{r}=(1,1,1)$. Calcule:
\begin{enumerate}
  \item El potencial eléctrico $V(\vec{r})$  en la posición $\vec{r}$ y la
    energía potencial eléctrica $U_e$ de esta configuración.
  \item El vector campo eléctrico $\vec{E}(\vec{r})$ en la posición $\vec{r}$ y
    la fuerza que actúa sobre el cuerpo de prueba.
  \item La aceleración que experimenta el cuerpo
\end{enumerate}

\item {\bf{Energía y Fuerzas}}

Aplique el desarrollo genérico visto en la clase 02-07 al caso de la interacción eléctrica, y demuestre que considerando sólo las interacciones gravitatorias y eléctricas, la segunda ley de Newton se escribe cómo
\[
  m \vec{a} = \vec{F_g} + \vec{F_e},
\]
donde $\vec{F_g}(\vec r) = G\frac{m_1 m_2}{|\vec{r}|^2} \hat{r}$ y $\vec{F_e}(\vec r) = k_e \frac{ q_1 q_2}{|\vec{r}|^2} \hat{r}$.

\item {\bf{Configuración de cargas I}}

Imagine la siguiente configuración de cuatro cargas:
\begin{center}
\begin{tabular}{ccc}
$i$ & $Q_i$ & $\vec{r}_i$ \\
\hline
$1$ & $+0.5$\,C & $(+1;0;0)$ \\
$2$ & $+1.5$\,C & $(-1;0;0)$ \\
$3$ & $-2.4$\,C & $(0;+1;0)$ \\
$4$ & $+0.4$\,C & $(0;-1;0)$ \\
\hline
\end{tabular}
\end{center}

\begin{enumerate}
\item Calcule lo siguiente: 

\begin{enumerate}
\item la carga neta del sistema;
\item la energía potencial eléctrica de esta configuración de cuatro cargas;
\item el potencial y el campo eléctrico en $\vec{r}=(0;0;0)$;
\item el potencial y el campo eléctrico en $\vec{r}=(1;1;1)$;
\item el potencial y el campo eléctrico en $\vec{r}=(200;200;200)$;
\item explique la razón del resultado obtenido en el punto anterior
\end{enumerate}

\item Ahora, suponga que un agente externo trae una carga de prueba $q=+0.7$\,C desde el infinito hasta la posición $\vec{r_i}=(0.5;1;1)$. Calcule:
\begin{enumerate}
\item el potencial y el campo eléctrico en ese punto;
\item la energía eléctrica de la configuración obtenida y el vector fuerza $\vec{F}$ que aparece sobre la carga de prueba en ese punto;
\item el trabajo realizado por el agente externo.
\end{enumerate}

\item Finalmente, suponga que el agente externo desplaza la carga de prueba $q$ desde $\vec{r_i}=(0.5;1;1)$ hasta $\vec{r_f}=(-0.5;-1;-1)$. Calcule: 
\begin{enumerate}
\item el potencial y el campo eléctrico en ese nuevo punto;
\item la energía eléctrica de la nueva configuración y el vector fuerza $\vec{F}$ que aparece sobre la carga de prueba en ese nuevo punto;
\item el trabajo realizado por el agente externo.
\end{enumerate}
\end{enumerate}

\item {\bf{Configuración de cargas II}}

Imagine la siguiente configuración de dos cargas:
\begin{center}
\begin{tabular}{ccc}
$i$ & $Q_i$ & $\vec{r}_i$ \\
\hline
$1$ & $+1.0$\,C & $(+0.1;0;0)$ \\
$2$ & $-1.0$\,C & $(-0.1;0;0)$,\\
\hline
\end{tabular}
\end{center}
esto es lo que se conoce como un dipolo eléctrico. Calcule:

\begin{enumerate}
\item la carga neta del sistema;
\item el potencial y el campo eléctrico en $\vec{r}=(0;0;0)$;
\item el potencial y el campo eléctrico en $\vec{r}=(0;1;0)$;
\item el potencial y el campo eléctrico en $\vec{r}=(50;50;50)$;
\end{enumerate}

\end{enumerate}
\end{document}
%%%%
