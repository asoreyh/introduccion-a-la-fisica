\documentclass[a4paper,12pt]{article}
\usepackage[spanish]{babel}
\hyphenation{co-rres-pon-dien-te}
%\usepackage[latin1]{inputenc}
\usepackage[utf8]{inputenc}
\usepackage[T1]{fontenc}
\usepackage{graphicx}
\usepackage{amsmath}

\usepackage[pdftex,colorlinks=true, pdfstartview=FitH, linkcolor=blue, citecolor=blue, urlcolor=blue, pdfpagemode=UseOutlines, pdfauthor={H. Asorey}, pdftitle={Introducción a la Física - Guía Auxiliar} pdfkeywords={pre-calculo}]{hyperref}
\usepackage[adobe-utopia]{mathdesign}

\hoffset -1.23cm
\textwidth 16.5cm
\voffset -2.0cm
\textheight 26.0cm

\begin{document}
\begin{center}
  {\small{Universidad Industrial de Santander - Escuela de Física}}\\
  {\bf{Introducción a la Física (Asorey-Sarmiento-Pinilla)}}\\
  \vspace{0.4cm}
  Guía Auxiliar (Fuera del programa del Curso) - Pre-cálculo \\ 2014
\end{center}

\renewcommand{\labelenumi}{\arabic{enumi})}
\renewcommand{\labelenumii}{\arabic{enumii})}


\begin{enumerate}
  \item {\bf{Factorizar}} %% http://facultad.bayamon.inter.edu/ntoro/gemafacto.htm
    \begin{enumerate}
      \item $a^2b - ab^2 = $
      \item $6p^2q + 24pq^2 = $
      \item $12x^3y - 48x^2y^2 = $
      \item $9m^2n + 18 mn^2 - 27mn= $
      \item $\frac14 ma + \frac14 mb + \frac14 mc = $
      \item $\frac15 x^3 + \frac{1}{10}x^2-\frac{1}{15}x = $
      \item $ x^2 - 8 x + 16 = $
      \item $ 16y^2 + 24 y + 9 = $
      \item $ 36a^2 - 12a + 1 = $
      \item $4x^2 + 20xy + 25y^2 = $
      \item $16x^2 - 25y^2 = $
      \item $144 - x^2y^2 = $
      \item $36 - 25a^2 = $
      \item $25 - 4a^2 = $
      \item $16m^2n^2 - 9p^2 = $
      \item $x^2 - 4x + 3 = $
      \item $x^2 - 2x - 15 = $
      \item $x^2 - 7xy - 18y^2 = $ 
      \item $12 - 4x - x^2 = $
      \item $5x^2 - 11x + 2 = $
      \item $6x^2 - 7x - 5 = $
      \item $12x^2 + 17x - 5 = $
      \item $7u^4 - 7u^2v^2 = $
      \item $kx^3 + 2kx^2 - 63kx = $
      \item $5x^3 - 55x^2 + 140x = $
      \item $4m^2n^2 + 24m^2n - 28m^2 = $
      \item $7hkx^2 + 21 hkx + 14hk = $
      \item $wx^2y - 9wxy + 14wy = $
      \item $2x^3 + 10x^2 + x + 5 = $ 
      \item $px + py + qx + qy = $
      \item $3x^3 + 12x^2 - 2x - 8 = $
      \item $3x^3 + 2x^2 + 12x + 8 = $
      \item $x^3 - 27 = $
      \item $125x^3 + y^3 = $
      \item $8y^3 + z^3 = $
      \item $64 - y^3 = $
    \end{enumerate}

  \item {\bf{Resolver las siguientes ecuaciones lineales}} %% http://www.acienciasgalilei.com/mat/problemas/ejerc1mat-ecuaciones-1.htm
    \begin{enumerate}
      \item $3x + 5 = 3 - 2x$
      \item $3x - 2(x + 1) = 2(3x - 1) + 4$
      \item $3(1 - 2x) - 4(1 - x) = x - 2(1 + x)$
      \item $\frac{x-1}{2} = \frac{2-x}{3}$
      \item $\frac{2(x-2)}{3} + \frac{3(1-x)}{2} = 1$
      \item $\frac{2(2-x)}{5} - \frac{3(2x-3)}{2} = \frac{4(1-x)}{3} + 2$
      \item $\frac{2x}{3} + \frac{3x}{2} = 2 (1 - 2x) - x$
      \item $2(2-x) + \frac{x}{3} - \frac{x}{2} = \frac{3(x+2)}{2}$
      \item $\frac23 \frac{1-x}{5} - \frac14 \frac{2x+3}{2}=\frac{x}{2}$
      \item $\frac{x}{2} + \frac{x}{3} - \frac{2x}{5} = x - \frac{4x}{3} - \frac{2(x+1)}{3}$
      \item $2 \left [ (1 - x) + 2 (2x - 4)) \right ] = \frac{x}{2} - 4$
      \item $\frac12 \frac{x-3}{3} = 1 - \frac{x}{4}$
      \item $- (2x + 4) - (3x - 1) = 3(x + \frac14 )$
      \item $p(x - 2) + 3x - p = 1 - x$
      \item $a(x - b) = c$
      \item $a(bx - c) = a(x - a)$
      \item $\frac{x-a}{b}=\frac{x+b}{a}$
      \item $\frac{ax-b}{c} = c (bx+a)$
    \end{enumerate}
  \item {\bf{Resolver las siguientes ecuaciones cuadráticas}} %% http://www.acienciasgalilei.com/mat/problemas/ejerc1mat-ecuaciones-1.htm
    \begin{enumerate}
      \item $x(x-1)=0$
      \item $x^2-2x=0$
      \item $x^2 - 4x = 0$
      \item $4x^2 - 16 = 0$
      \item $4x^2 + 16 = 0$
      \item $2x^2 - 8 = 0$
      \item $2x^2 + 4x = 0$
      \item $x^2 - 5x + 6 = 0$
      \item $x^2 + x - 6 = 0$
      \item $2x^2 + 2x - 12 = 0$
      \item $- 2x^2 - 2x + 4 = 0$
      \item $3x^2 - 9x - 12 = 0$
      \item $x^2  + x + 1 = 0$
      \item $x^2 + 5x + 6 = 0$
      \item $2x^2 + 10x + 12 = 0$
      \item $x^2 - 2x - 3 = 0$
      \item $-x^2 + 2x + 3 = 0$ 
      \item $x^2 - 6x + 5 = 0$
      \item $x^2 + x - 12 = 0$
    \end{enumerate}

  \item {\bf{Responda y resuelva según corresponda}} %% http://www.vitutor.com/algebra/sistemas%20I/sg_e.html
    \begin{enumerate}
      \item Decir si son verdaderas o falsas las siguientes afirmaciones:
        \begin{enumerate}
          \item En un sistema compatible indeterminado se puede eliminar una ecuación y obtener un sistema equivalente.
          \item Un sistema compatible indeterminado es equivalente a un sistema homogéneo.
          \item Todo sistema compatible indeterminado tiene dos ecuaciones iguales.
          \item De un sistema incompatible podemos extraer otro compatible (no equivalente) eliminando ecuaciones.
        \end{enumerate}
    \item 
      \[
        \begin{cases}
          x + y + z = 1\\
          2x + 3y - 4z = 9 \\
          x - y + z = -1
        \end{cases}
      \]
    \item
      \[
        \begin{cases}
          3x + 2y + z = 1\\
          5x + 3y + 4z = 2 \\
          x + y - z = 1
        \end{cases}
      \]
    \item 
      \[
        \begin{cases}
          x - 9y + 5z = 33\\
          x + 3y - z = -9\\
          x - y + z = 5
        \end{cases}
      \]
    \item 
      \[
        \begin{cases}
          x + y - z = 1\\
          3x + 2y + z = 1\\
          5x + 3y + 4z = 2\\
          -2x-y+5z = 6
        \end{cases}
      \]
    \item ¿Existe algún valor de $m$ para el cual el sistema es compatible? Si lo hay, resolver el sistema para ese valor de $m$
      \[
        \begin{cases}
          x + my + z = 1\\
          mx + y + (m-1) z = m\\
          x + y + z = m+1
        \end{cases}
      \]
    \item El dueño de un bar ha comprado refrescos, cerveza y vino por importe de \$ $500000$ (sin impuestos). El valor del vino es \$ $60000$ menos que el de los refrescos y de la cerveza conjuntamente. Teniendo en cuenta que los refrescos deben pagar un IVA del $6\%$, por la cerveza del $12\%$ y por el vino del $30\%$, lo que hace que la factura total con impuestos sea de \$ $592400$, calcular la cantidad invertida en cada tipo de bebida.
    \item Una empresa tiene tres minas con menas de composiciones:\\
      \begin{center}
        \begin{tabular}{cccc}
          Mina & Níquel (\%) & Cobre (\%) & Hierro (\%) \\
          A & 1 & 2 & 3 \\
          B & 2 & 5 & 7 \\
          C & 1 & 3 & 1 \\
        \end{tabular}
      \end{center}
      ¿Cuántas toneladas de cada mina deben utilizarse para obtener $7$ toneladas de Níquel, $18$ de Cobre y $16$ de Hierro?
    \item La edad de un padre es doble de la suma de las edades de sus dos hijos, mientras que hace unos años (exactamente la diferencia de las edades actuales de los hijos), la edad del padre era triple que la suma de las edades, en aquel tiempo, de sus hijos. Cuando pasen tantos años como la suma de las edades actuales de los hijos, la suma de edades de las tres personas será $150$ años. ¿Qué edad tenáa el padre en el momento de nacer sus hijos?
    \item Se venden tres especies de cereales: trigo, cebada y mijo. Cada volumen de trigo se vende por \$ $4000$, el de la cebada por \$ $2000$ y el de mijo por \$ $500$. Si se vende $100$ volúmenes en total y si obtiene por la venta \$ $100000$, ¿cuántos volúmenes de cada especie se venden?
    \item Se tienen tres lingotes compuestos del siguiente modo:
      \begin{itemize}
        \item El primero de $20$\,g de oro, $30$\,g de plata y $40$\,g de cobre.
        \item El segundo de $30$\,g de oro, $40$\,g de plata y $50$\,g de cobre.
        \item El tercero de $40$\,g de oro, $50$\,g de plata y $90$\,g de cobre.
      \end{itemize}
      ¿Qué peso habrá de tomarse de cada uno de los lingotes anteriores para formar un nuevo lingote de $34$\,g de oro, $46$\,g de plata y $67$\,g de cobre?
    \end{enumerate}
  \end{enumerate}
\end{document}
